\documentclass[12pt]{article}
\usepackage{setspace}
\setstretch{1}
\usepackage{amsmath,amssymb, amsthm}
\usepackage{graphicx}
\usepackage{bm}
\usepackage[hang, flushmargin]{footmisc}
\usepackage[colorlinks=true]{hyperref}
\usepackage[nameinlink]{cleveref}
\usepackage{footnotebackref}
\usepackage{url}
\usepackage{listings}
\usepackage[most]{tcolorbox}
\usepackage{inconsolata}
\usepackage[papersize={8.5in,11in}, margin=1in]{geometry}
\usepackage{float}
\usepackage{caption}
\usepackage{esint}
\usepackage{url}
\usepackage{enumitem}
\usepackage{subfig}
\usepackage{wasysym}
\newcommand{\ilc}{\texttt}
\usepackage{etoolbox}
\usepackage{algorithm}
% \usepackage{algorithmic}
\usepackage[noend]{algpseudocode}
\usepackage{tikz}
\usetikzlibrary{matrix,positioning,arrows.meta,arrows}
\patchcmd{\thebibliography}{\section*{\refname}}{}{}{}

\makeatletter
\renewcommand{\@seccntformat}[1]{}
\makeatother


\begin{document}



\title{\textbf{EECS 325: Assignment 4}}

\author{Shaochen (Henry) ZHONG, \ilc{sxz517} }
\date{Due and submitted on 04/30/2020 \\ EECS 325, Dr. WANG}
\maketitle

% % % % % % % % % % % % % % % % % % % % % % % % % % % % % % % % % %
% % % % % % % % % % % % % % % % % % % % % % % % % % % % % % % % % %
% % % % % % % % % % % % % % % % % % % % % % % % % % % % % % % % % %
\section{Question 1}

% % % % % % % % % % % % % % % % % % % % % % % % % % % % % % % % % %
\subsection{(a)}
The second best path from $E$ to $F$ is via neighbor $B$ ($E \rightarrow B \rightarrow E \rightarrow F$, with a cost of $5$).

(If reused of ``banned'' route $E \rightarrow F$ is now allowed, then the second best path is $E \rightarrow D \rightarrow F$ with a cost of $6$. However it is not explicitly restricted by the question so I guess $B$ is still the best neighbor to go with. I am providing this alternative solution just in case...)

% % % % % % % % % % % % % % % % % % % % % % % % % % % % % % % % % %
\subsection{(b)}
$E$'s advertised distance to $F$ is $6$.

\begin{table}[H]
\center
\begin{tabular}{cc}
Route  & \begin{tabular}[c]{@{}c@{}}After E's update\\ (distance, next-hop)\end{tabular} \\
A to F & 5, D                                                                            \\
B to F & 5, C                                                                            \\
D to F & 3, F
\end{tabular}
\end{table}


% % % % % % % % % % % % % % % % % % % % % % % % % % % % % % % % % %
\subsection{(c)}
$B$'s advertised distance to $F$ is $6$.

\begin{table}[H]
\center
\begin{tabular}{cc}
Route  & \begin{tabular}[c]{@{}c@{}}After B's update\\ (distance, next-hop)\end{tabular} \\
A to F & 5, D                                                                            \\
C to F & 5, F                                                                            \\
E to F & 6, D/F
\end{tabular}
\end{table}

% % % % % % % % % % % % % % % % % % % % % % % % % % % % % % % % % %
\subsection{(d)}
$C$'s advertised distance to $F$ is $5$.

\begin{table}[H]
\center
\begin{tabular}{cc}
Route  & \begin{tabular}[c]{@{}c@{}}After C's update\\ (distance, next-hop)\end{tabular} \\
B to F & 6, C
\end{tabular}
\end{table}

Now all routers are following the correct shortest path, as in the next iteration (4th) the table will not update anymore.

% % % % % % % % % % % % % % % % % % % % % % % % % % % % % % % % % %
\subsection{(e)}
\begin{enumerate}[label=(\arabic*)]
    \item 1st iteration, update $E$ to $F$ as $6, F$.
    \item 2nd iteration, fill in $A$ to $F$, $B$ to $F$, update $C$ to $F$ as $5, F$, update $E$ to $F$ as $6, D/F$.
    \item 3rd iteration, update $B$ to $F$ as $6, C$.
    \item 4th iteration, no more update.
\end{enumerate}

Thus the tables will \textbf{NOT} converge faster with \textit{poisoned reverse}, as either implemented or not, both approaches took 4 iterations to converge.

% % % % % % % % % % % % % % % % % % % % % % % % % % % % % % % % % %
% % % % % % % % % % % % % % % % % % % % % % % % % % % % % % % % % %
% % % % % % % % % % % % % % % % % % % % % % % % % % % % % % % % % %
\section{Question 2}

% % % % % % % % % % % % % % % % % % % % % % % % % % % % % % % % % %
\begin{enumerate}[label=(\alph*)]
    \item \ilc{eBGP}, 4c tells.
    \item \ilc{iBGP}, 3b tells.
    \item \ilc{eBGP}, 3a tells.
    \item \ilc{iBGP}, 1a or 1b tells.
\end{enumerate}

\smallskip

\begin{enumerate}[label=(\roman*)]
    \item \ilc{A, A-C, A-C-F, A-D, A-D-G}
    \item \ilc{C, C-F}
    \item \ilc{E}
    \item \ilc{F}
\end{enumerate}


% \section{References}
%
% \nocite{*}
% \raggedright
% \bibliography{references.bib}
% \bibliographystyle{plain}


\end{document}