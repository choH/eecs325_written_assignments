\documentclass[12pt]{article}
\usepackage{setspace}
\setstretch{1}
\usepackage{amsmath,amssymb, amsthm}
\usepackage{graphicx}
\usepackage{bm}
\usepackage[hang, flushmargin]{footmisc}
\usepackage[colorlinks=true]{hyperref}
\usepackage[nameinlink]{cleveref}
\usepackage{footnotebackref}
\usepackage{url}
\usepackage{listings}
\usepackage[most]{tcolorbox}
\usepackage{inconsolata}
\usepackage[papersize={8.5in,11in}, margin=1in]{geometry}
\usepackage{float}
\usepackage{caption}
\usepackage{esint}
\usepackage{url}
\usepackage{enumitem}
\usepackage{subfig}
\usepackage{wasysym}
\newcommand{\ilc}{\texttt}
\usepackage{etoolbox}
\usepackage{algorithm}
% \usepackage{algorithmic}
\usepackage[noend]{algpseudocode}
\usepackage{tikz}
\usetikzlibrary{matrix,positioning,arrows.meta,arrows}
\patchcmd{\thebibliography}{\section*{\refname}}{}{}{}

\makeatletter
\renewcommand{\@seccntformat}[1]{}
\makeatother


\begin{document}



\title{\textbf{EECS 325: Take Home Final}}

\author{Shaochen (Henry) ZHONG, \ilc{sxz517} }
\date{Due and submitted on 05/01/2020 \\ EECS 325, Dr. WANG}
\maketitle

% % % % % % % % % % % % % % % % % % % % % % % % % % % % % % % % % %
% % % % % % % % % % % % % % % % % % % % % % % % % % % % % % % % % %
% % % % % % % % % % % % % % % % % % % % % % % % % % % % % % % % % %
\section{Single Choice Questions}


\begin{enumerate}
    \item
    \begin{enumerate}[label=(\roman*)]
        \item c
        \item b
        \item a
        \item d
    \end{enumerate}
    \item d
    \item c
    \item c
    \item e
\end{enumerate}


% % % % % % % % % % % % % % % % % % % % % % % % % % % % % % % % % %
% % % % % % % % % % % % % % % % % % % % % % % % % % % % % % % % % %
% % % % % % % % % % % % % % % % % % % % % % % % % % % % % % % % % %
\section{Short Answer Questions}

% % % % % % % % % % % % % % % % % % % % % % % % % % % % % % % % % %
\subsection{1}

It will operate between Alice's email client and Alice's outgoing mail server; also Alice's outgoing mail server and Bob's incoming mail server.

% % % % % % % % % % % % % % % % % % % % % % % % % % % % % % % % % %
\subsection{2}
IGP is for intra-AS routing, means navigating packets within the same AS; BGP is for inter-AS routing, means navigating packets across different AS.

% % % % % % % % % % % % % % % % % % % % % % % % % % % % % % % % % %
\subsection{3}

When \ilc{cwnd} exceeds \ilc{ssthresh} ($\frac{1}{2}$ of current \ilc{cwnd}).

\noindent It is possible, when a timeout occurs or triple duplicate \ilc{ACK} is received (as some packets are getting through).

% % % % % % % % % % % % % % % % % % % % % % % % % % % % % % % % % %
\subsection{4}

Yes.

\noindent A router's number of IP Addresses is equal to the number of interface it has to connect to different subnets. There will be no same IP Address used for multiple different subnets.


% % % % % % % % % % % % % % % % % % % % % % % % % % % % % % % % % %
% % % % % % % % % % % % % % % % % % % % % % % % % % % % % % % % % %
% % % % % % % % % % % % % % % % % % % % % % % % % % % % % % % % % %
\section{IP Fragmentation}

$\frac{2400 - 20}{700 - 20} = 3.5 \approx 4$. Thus, 4 fragmentations are generated.


\begin{table}[H]
\centering
\begin{tabular}{cccc}
\textbf{Seg Num} & \textbf{\begin{tabular}[c]{@{}c@{}}Length\\ (bytes)\end{tabular}} & \textbf{Flagflag} & \textbf{Offset} \\
1                & 700                                                               & 1                 & 0               \\
2                & 700                                                               & 1                 & 85              \\
3                & 700                                                               & 1                 & 170             \\
4                & 360                                                               & 0                 & 255
\end{tabular}
\end{table}



% % % % % % % % % % % % % % % % % % % % % % % % % % % % % % % % % %
% % % % % % % % % % % % % % % % % % % % % % % % % % % % % % % % % %
% % % % % % % % % % % % % % % % % % % % % % % % % % % % % % % % % %
\section{IP Addressing}

\begin{itemize}
    \item subnet 1: \ilc{223.1.13.0/25}
    \item subnet 2: \ilc{223.1.13.128/26}
    \item subnet 3: \ilc{223.1.13.192/26}
\end{itemize}

% \section{References}
%
% \nocite{*}
% \raggedright
% \bibliography{references.bib}
% \bibliographystyle{plain}

% % % % % % % % % % % % % % % % % % % % % % % % % % % % % % % % % %
% % % % % % % % % % % % % % % % % % % % % % % % % % % % % % % % % %
% % % % % % % % % % % % % % % % % % % % % % % % % % % % % % % % % %
\section{Routing Algorithm}


\begin{table}[H]
\centering
\begin{tabular}{ccccccc}
\textbf{S} & \textbf{N'} & \textbf{D(v),  p(v)} & \textbf{D(w), p(w)} & \textbf{D(x), p(x)} & \textbf{D(y), p(y)} & \textbf{D(z), p(z)} \\
0          & u           & 2, u                 & $\infty$                 & 1, u                & $\infty$                 & $\infty$                 \\
1          & ux          & 2, u                 & 4, x                & 1, u                & 2, x                & $\infty$                 \\
2          & uxy         & 2, u                 & 3, y                & 1, u                & 2, x                & 4, y                \\
3          & uxyv        & 2, u                 & 3, y                & 1, u                & 2, x                & 4, y                \\
4          & uxyvw       & 2, u                 & 3, y                & 1, u                & 2, x                & 4, y                \\
5          & uxyvwz      & 2, u                 & 3, y                & 1, u                & 2, x                & 4, y
\end{tabular}
\end{table}

The least-cost path is: $u \rightarrow x  \rightarrow y  \rightarrow z $


% % % % % % % % % % % % % % % % % % % % % % % % % % % % % % % % % %
% % % % % % % % % % % % % % % % % % % % % % % % % % % % % % % % % %
% % % % % % % % % % % % % % % % % % % % % % % % % % % % % % % % % %
\section{Data Link Layer}

% % % % % % % % % % % % % % % % % % % % % % % % % % % % % % % % % %
\subsection{(a)}

Known that $G = 10101$ and $r = 4$, now we have $\frac{D \cdot 2^r}{G} = \frac{110110101110000}{10101} = ... 0011$

\noindent The CRC code will be \ilc{110110101110011}.

% % % % % % % % % % % % % % % % % % % % % % % % % % % % % % % % % %
\subsection{(b)}
No. Let the error be \ilc{10101}, we have $F(x) = 110110101110011$ and $G'(X) = x^4+x^2+1$. Then we must have $\frac{F(x)+G'(x)}{G(x)} = \frac{G'(x)}{G(x)} = ... 0$ in terms of the reminder. Thus, it cannot detect odd number of bit errors.

% % % % % % % % % % % % % % % % % % % % % % % % % % % % % % % % % %
% % % % % % % % % % % % % % % % % % % % % % % % % % % % % % % % % %
% % % % % % % % % % % % % % % % % % % % % % % % % % % % % % % % % %
\section{Multiple Access Protocol}

For $B$ to be success at any slot we have $P(B) = p(1-p)^2$ since $A$ and $C$ should not be success at this slot.
Thus, for $B$ to be success for the first time in $S_4$, we have $(1-P(B))^3 \cdot P(B) = (p(1-p)^2)^3 \cdot p(1-p)^2 = (1-p(1-p)^2)^3 \cdot p(1-p)^2$.


% % % % % % % % % % % % % % % % % % % % % % % % % % % % % % % % % %
% % % % % % % % % % % % % % % % % % % % % % % % % % % % % % % % % %
% % % % % % % % % % % % % % % % % % % % % % % % % % % % % % % % % %
\section{Course Evaluation}

(b)

\end{document}