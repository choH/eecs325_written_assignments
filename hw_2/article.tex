\documentclass[12pt]{article}
\usepackage{setspace}
\setstretch{1}
\usepackage{amsmath,amssymb, amsthm}
\usepackage{graphicx}
\usepackage{bm}
\usepackage[hang, flushmargin]{footmisc}
\usepackage[colorlinks=true]{hyperref}
\usepackage[nameinlink]{cleveref}
\usepackage{footnotebackref}
\usepackage{url}
\usepackage{listings}
\usepackage[most]{tcolorbox}
\usepackage{inconsolata}
\usepackage[papersize={8.5in,11in}, margin=1in]{geometry}
\usepackage{float}
\usepackage{caption}
\usepackage{esint}
\usepackage{url}
\usepackage{enumitem}
\usepackage{subfig}
\usepackage{wasysym}
\newcommand{\inlinecode}{\texttt}
\usepackage{etoolbox}
\usepackage{algorithm}
% \usepackage{algorithmic}
\usepackage[noend]{algpseudocode}
\usepackage{tikz}
\usetikzlibrary{matrix,positioning,arrows.meta,arrows}
\patchcmd{\thebibliography}{\section*{\refname}}{}{}{}

\makeatletter
\renewcommand{\@seccntformat}[1]{}
\makeatother


\begin{document}



\title{\textbf{EECS 325: Assignment 2}}

\author{Shaochen (Henry) ZHONG, \inlinecode{sxz517} }
\date{Due and submitted on 02/20/2020 \\ EECS 325, Dr. WANG}
\maketitle

\section{Question 1}

\subsection{a.}

\begin{align*}
    T(\text{base file}) &= \frac{10 \text{KB}}{10 \text{Mbps}} = \frac{10 \cdot 8 \text{Kb}}{10,000 \text{Kbps}} = 0.008 \text{s} \\
    T(\text{object}) &= \frac{100 \text{KB}}{10 \text{Mbps}} = \frac{100 \cdot 8 \text{Kb}}{10,000 \text{Kbps}} = 0.08 \text{s}
\end{align*}


\begin{align*}
    T(\text{non-persistent 1 TCP}) &= 2 \cdot \text{RTT} + T(\text{base file}) + 20 \cdot (2 \cdot \text{RTT} + T(\text{object})) \\
    &= 2 \cdot 0.1s + 0.008s + 20 (2 \cdot 0.1s + 0.008s) \\
    &= 5.808s
\end{align*}


\subsection{b.}


\begin{align*}
    C(\text{Bandwith per connection}) &= \frac{10 \text{Mbps}}{4} = \frac{10,000 \text{Kbps}}{4} = 2,500 \text{Kbps} \\
    \Longrightarrow T(\text{object per connection}) &= \frac{100 \cdot 8 \text{Kb}}{2,500 \text{Kbps}} = 0.32 \text{s}
\end{align*}

\begin{align*}
    T(\text{non-persistent 4 TCP}) &= 2 \cdot \text{RTT} + T(\text{base file}) + 20 \cdot \frac{1}{4} \cdot (2 \cdot \text{RTT} + T(\text{object per connection})) \\
    &= 2 \cdot 0.1s + 0.008s + \frac{20}{4} (2 \cdot 0.1s + 0.32s) \\
    &= 2.808s
\end{align*}


\subsection{c.}

\begin{align*}
    T(\text{persistent pipelined}) &= 2 \cdot \text{RTT} + T(\text{base file}) + RTT(\text{request}) + 20 \cdot T(\text{object}) \\
    &= 0.2s + 0.008s + 0.1s + 20(0.08s) \\
    &= 1.908s
\end{align*}

\subsection{d.}

\begin{align*}
    C(\text{Bandwith per connection}) &= \frac{10 \text{Mbps}}{2} = \frac{10,000 \text{Kbps}}{2} = 5,000 \text{Kbps}
\end{align*}


\begin{align*}
    T(\text{persistent pipelined 2 TCP}) &= 2 \cdot \text{RTT} + T(\text{base file})  \\
    &+ \frac{\text{number of objects}}{number of connections} \cdot (RTT(\text{request}) + T(\text{object})) \\
    &= 0.2s + 0.008s + \frac{20}{2} \cdot (0.1s + 0.16s) \\
    &= 2.808s
\end{align*}

% % % % % % % % % % % % % % % % % % % % % % % % % % % % % % % % % %
% % % % % % % % % % % % % % % % % % % % % % % % % % % % % % % % % %
% % % % % % % % % % % % % % % % % % % % % % % % % % % % % % % % % %

\section{Question 2}

\subsection{a.}

\begin{align*}
    D &= \frac{\text{avergae object size}}{\text{access link capacity}} = \frac{S}{C}\ \ \ \text{s}\\
    \text{Average access delay} &= \frac{D}{1-DB} = \frac{\frac{S}{C}}{1-\frac{S}{C} \cdot A} = \frac{S}{C-SA}\ \ \ \text{s} \\
    T_1 &= \text{Average access delay} + T = \frac{S}{C-SA} + T \ \ \ \text{s}
\end{align*}


\subsection{b.}

\begin{align*}
    \text{For} \ B &= (1-p) \cdot A \\
    \text{Average access delay} &= \frac{D}{1-DB} = \frac{\frac{S}{C}}{1-(\frac{S}{C} \cdot (1-p) A)} = \frac{S}{C-(1-p) \cdot SA}\ \ \ \text{s} \\
    T_2 &= \text{Average access delay} + T = \frac{S}{C-(1-p) \cdot SA} + T \ \ \ \text{s}
\end{align*}

\subsection{c.}
\subsubsection{c.(i)}
The total time will be $0$ second if the requested information is cached in the local DNS server. Otherwise, it willl be $R_1 + R_2 + R_3$ as the DNS resolver will have to query from the root, TLD, and authoritative DNS server.


\subsubsection{c.(ii)}
If the DNS solver may query the requested information from its ISP, the additional $R_0$ will be added to the delay. If not, the ISP will again have to go through the process of query from root, TLD, and authoritative DNS server, thus cause a delay of $R_0 + R_1 + R_2 + R_3$.

\section{Question 3}

% \section{References}
%
% \nocite{*}
% \raggedright
% \bibliography{references.bib}
% \bibliographystyle{plain}


\end{document}